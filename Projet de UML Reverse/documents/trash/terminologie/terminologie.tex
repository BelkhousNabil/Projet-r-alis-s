\documentclass[hidelinks, 10pt,a4paper]{article}

\usepackage[english,francais]{babel}
\usepackage[utf8]{inputenc}
\usepackage{geometry}
\usepackage[T1]{fontenc}
\usepackage[pdftex]{graphicx}
\usepackage{adjustbox}
\usepackage{color}
\usepackage{setspace}
\usepackage{hyperref}
\usepackage[french]{varioref}
\usepackage{comment}


\usepackage{fancyhdr}
\pagestyle{fancy}

\renewcommand{\headrulewidth}{1pt}
\fancyhead[L]{}
\fancyhead[C]{\textbf{UML Reverse}}
\fancyhead[R]{\includegraphics[width=2cm]{imgimgTerminologie/universite-rouen.jpg}}



%Opening
\title{\bfseries Terminologie\\Projet UML Reverse}
\geometry{hmargin=2.5cm,vmargin=3cm}


\begin{document}
\maketitle
\begin{center}
\begin{tabular}{ll}
  Version~: & 0.1\\[.5em]
  Date~: & \date{\today}\\[.5em]
  Rédigé par~: & Nabil \textsc{Belkhous}\\
               & Stephen \textsc{Cauchois}\\
               & Anthony \textsc{Godin}\\
               & Yohann \textsc{Henry}\\
               & Florian \textsc{Inchingolo}\\
               & Nicolas \textsc{Meniel}\\
               & Saad \textsc{Mrabet}\\[.5em]

\end{tabular}
\end{center}

\newpage

\begin{description}
  \item[Agilité]~\\ Capacité à favoriser le changement et à y répondre en vue de s'adapter au mieux à un environnement turbulent.
  \item[Cacher/Afficher (action)]~\\ Cacher ou afficher un élément. Un élément caché ne sera pas généré dans toute exportation. Cette action est enregistrée 
	dans le fichier de paramètres.
	 \item[\textbf{Chef de projet}]~\\ Responsable de gérer et d'unir le travail des membres de l'équipe pour aboutir à la réalisation du projet.
 \item[Créer (action)]~\\ Dans le cas d’un élément, créer et ajouter l’élément dans le diagramme. Dans le cas de la création d’un diagramme, ajouter un nouveau diagramme 
  vide ou généré au projet.
   \item[Déplacer (action)]~\\ Déplacer un élément graphiquement. La position de l’élement est enregistrée dans le fichier de paramètres. Le déplacement d'une entité 
  se fait aussi de façon visuelle. Ses éléments doivent le suivre et les relations aussi.
    \item[Diagramme]~\\ Dans ce document, quand on parle de diagramme, on ne fait pas référence simplement à un dessin mais à l’ensemble 
			des fichiers qui permettent de définir ce diagramme. On les édite dans un projet grâce à un fichier écrit en PlantUML 
			et pouvant être «~stylisés~» par un fichier de paramètres.
   \item[Éditer (action)]~\\ Suppose l’ouverture (le chargement), la modification, la sauvegarde et la suppression.
  \item[Élément]~\\ Une des informations affichées dans un diagramme. Par exemple, une entité est un élément, une méthode est un élément, une fléche est un élément.
  \item[Entité]~\\ Élément de base d’un diagramme. Il n’a pas d’élément parent. Dans le cas d’un diagramme de classes, par exemple, une classe et une note sont des
      entités, mais une méthode est un élément de l’entité Classe.
    \item[\textbf{L'érgonomie}]~\\ L'érgonomie pour ce produit consiste à l'utilisateur de pouvoir rapidement faire ce qu'il veut sans perdre de temps à 
	  manipuler le produit. On fixe une limite à 4 cliques maximum pour effectuer toutes les actions possibles.
	  Le produit doit pouvoir être utilisé facilement sans lire de manuel.
   \item[Fichier de paramètres]~\\ Spécifique à notre application, il est là pour enregistrer les modifications graphiques de l’utilisateur, comme la position des 
	  entités ou la liste des éléments cachés. Ces informations ne sont pas supportées par PlantUML. Ces fichiers sont associés à 
	  n’importe quel diagramme de notre application.
  \item[Flèche]~\\ Lien entre deux entités~; une relation ou une action, par exemple. Elle est représentée par une tête de flèche, une queue de flèche et un corps 
    éditables. Elle peut contenir jusqu’à 3 chaînes de caractères en fonction du diagramme.
  \item[Frontière]~\\ Une frontière se place dans un diagramme d’utilisation. Elle est représentée par un rectangle et délimite un système où on place généralement les cas 
    d’utilisation.
  \item[Itération]~\\ Appellé aussi sprint. (Voir la définition)
    \item[\textbf{JUnit}]~\\ Framework de test unitaire fonctionnant pour JAVA.
    \item[\textbf{Maven}]~\\ Outil pour la gestion et l'automatisation de production des projets logiciels JAVA.
    \item[Mélée]~\\ Réunion de planification « juste à temps » et permet aux développeurs de faire un point de coordination sur 
		    les tâches en cours et sur les difficultés rencontrées.
   \item[Modifier (action)]~\\ Modifier un élément ou l’un de ses contenus. Par exemple, si l’élément peut être nommé, nous pouvons modifier son nom.
   \item[Mot/Nom (dans un diagramme)]~\\ Suite de caractères en UTF-8 sans espace, pouvant avoir une taille d’au moins 100 octets.
    \item[\textbf{MVC}]~\\ Modèle Vue Controleur. Le modèle s'occupe de gérer la partie logique du projet et le controleur s'occupe
			de faire coopérer la modèle et la vue.
\item[\textbf{Phase de test}]~\\ Période où l'équipe test les fonctionnalités du projet en fonction du cahier des tests et les valide avant de rendre un livrable.
  \item[PlantUML]~\\ Projet open source qui permet de dessiner rapidement~:
    \begin{itemize}
      \item des diagrammes de séquence~;
      \item des diagrammes de cas d’utilisation~;
      \item des diagrammes de classes (et de paquetages)~;
      \item des diagrammes d’activités~;
      \item des diagrammes de composants~;
      \item des diagrammes d’états~;
      \item des diagrammes d’objets.
    \end{itemize}
    Les diagrammes sont définis à l’aide d’un pseudo-langage simple et intuitif. % http://PlantUML.sourceforge.net/PlantUML%20Language%20Reference%20Guide.pdf
    \item[\textbf{Preuve de concept}]~\\ Réalisation courte ou incomplète d'une certaine méthode ou idée pour démontrer sa faisabilité.
    \item[\textbf{Product Backlog}]~\\ L'ensemble des fonctionnalités du produit que l'on veut développer.
    \item[\textbf{Product Owner}]~\\  Représentant du client ou des utilisateurs.
     \item[\textbf{Projet}]~\\ Ensemble de diagrammes contenus dans un dossier de projet. Ce dossier peut actuellement contenir les sous dossiers : 
	\begin{itemize}
	  \item case
	  \item clas
	  \item seq
	\end{itemize}
	Les diagrammes sont rangés dans un de ces dossiers selon son type. Les diagrammes sont 
	enregistrés dans un dossier du même nom que le diagramme et contiennent les dossiers :
	\begin{itemize}
	  \item src/plantuml: contient le code source.
	  \item parameters/css: contient les fichiers de paramètres.
	\end{itemize}
	Il y a au moins un fichier source (dans src/plantuml) du même (strictement) nom que le projet. Il en est de même pour le fichier de paramètre. 
  \item[Reverse]~\\ Génération d’un diagramme UML à partir d’un code source. Dans notre application, seul un code source Java peut être utilisé en entrée.
  \item[Sauvegarder (action)]~\\ Sauvegarder un diagramme dans le projet avec ses paramètres.
  \item[Scrum]~\\ Cadre de développement informatique dit agile. Il est utilisé pour développer des produits en quelques mois. Formé de plusieurs itérations (sprint)
		  validées par le client, de mêlés organisées par l'équipe toutes les semaines et de l'écriture d'un product backlog.
  \item[Sprint]~\\ Bloc de temps fixé aboutissant à créer un incrément du produit potentiellement livrable.
  \item[Sprint Backlog]~\\ Le sprint backlog représente l’ensemble des tâches sélectionnées depuis le product backlog 
			  lors de la réunion de planification afin d’améliorer le produit avec de nouvelles fonctionnalités.
  \item[Sprint Rétrospective]~\\ Réunion dont le but est d'améliorer le processus pour le prochain sprint.
  \item[Sprint Review]~\\  L'objectif est de valider l'incrément de produit qui a été réalisé pendant le sprint. Cela permet une visibilité sur le produit.
			On analyse ce qu'on a fait du produit.
  \item[Supprimer (action)]~\\ Supprimer un élément et tous ses contenus.
  \item[UML]~\\ Langage de modélisation unifié, de l’anglais Unified Modeling Language. C’est un langage de modélisation 
    graphique à base de pictogrammes conçu pour fournir une méthode normalisée pour visualiser la conception d’un système. %wikipedia
  \item[Texte (dans un diagramme)]~\\ Texte en UTF-8 de taille raisonnable, pouvant avoir une taille d’au moins 1 Mo.
   \item[\textbf{Type de projet}]~\\ Le type d'un projet c'est le type de diagramme qu'il représente: cas d'utilisation ou classe ou séquence.
\end{description}
\end{document}
