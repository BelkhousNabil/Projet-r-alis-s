\documentclass[a4paper,10pt]{article}
\usepackage[utf8]{inputenc}
\usepackage{geometry} 	%pour les marges
\usepackage{color} %pour la definition de nouvelles couleurs
\usepackage{graphicx} %pour ajouter des images 
\usepackage[final]{pdfpages} 
\usepackage[francais]{babel} % ``Franciser '' le document.
\usepackage	
  [colorlinks=false, urlcolor=red, breaklinks, pagebackref, citebordercolor={0 0 0}, 
  filebordercolor={0 0 0}, linkbordercolor={0 0 0}, pagebordercolor={0 0 0}, 
  runbordercolor={0 0 0}, urlbordercolor={0 0 0}, pdfborder={0 0 0}]
	{hyperref} % Ajouter le package des lien redirigeant sans les encadrer 
\usepackage{eurosym}

%couleurs pour les morceaux de code
\definecolor{codegreen}{rgb}{0,0.6,0}
\definecolor{codered}{rgb}{1,0.1,0.2}
\definecolor{codegray}{rgb}{0.5,0.5,0.5}
\definecolor{codepurple}{rgb}{0.58,0,0.82}
\definecolor{backcolour}{rgb}{0.95,0.95,0.92}
 

%modificaion des marges
\geometry{hmargin=2.5cm,vmargin=3cm}


%opening
\title{Compte rendu des réunions : UML Reverse}

\author{
Florian INCHINGOLO\\ 
Yohann HENRY\\ 
Nabil BELKHOUS\\ 
Stephen CAUCHOIS\\ 
Saad MRABET\\ 
Nicolas MENIEL\\ 
Anthony GODIN}

\begin{document}
\maketitle
\tableofcontents
\newpage

\section{Réunions client}
\subsection{Vendredi 16/10/2015}
\underline{Présents: M.Hérauville avec toute l'équipe}
\\ \\
Réunion rapide avec le client qui nous a résumé ce qu'il voulait et ses disponibilités.
\subsection{Jeudi 23/10/2015}
\underline{Présents: M.Hérauville avec toute l'équipe}
\\ \\
Présisions plus techniques des envies du client et réponses à certaines questions.
\subsection{Jeudi 05/11/2015}
\underline{Présents: M.Hérauville - Anthony - Nabil}
\\ \\
\textbf{\underline{Questions :}}
\begin{enumerate}
  \item Doit-on créer un outil ou un Logiciel ? C'est à dire, quand nous créons du plantUML en lisant du code, doit on stocker toutes les infos lues dans
  le plantUML ou juste ce qui ait demandé par l'utilisateur ?\\
  Si on stock tous dans le plantUML, ça veut dire qu'on définira ce qu'on veut afficher sur nos diagrammes dans un fichier de parametre (surement un XML). 
  Mais ce choix impose qu'on perdera ces paramètres si on change de logiciel pour afficher notre diagramme. D'un autre coté avec nos logiciels on perd rien.
  (ps: les parametres dont je parle c'est genre: afficher ou non les fonctions privés, les attributs etc... Rien de plus)
  \item On suppose que le code java est valide à la compilation ? S'il ne l'air pas, que fait-on ?
  \item Le fichier de style doit t'il obligatoirement être un fichier css ? On pourrait sauver les parametres du diagramme dans un XML.
  \item Le client a t'il des contraintes techniques ? S'interesse t'il a la structure technique du projet ?
  \item Peut on utiliser le code d'un outil sous licence EPL ?
\end{enumerate}

\textbf{\underline{Réponses :}} 
\begin{enumerate}
 \item Il serait dommage de perdre des infos dans le plantUML. On garde le plantUML complet et on cache les info grace à un autre fichier. 
	Ou on peut avoir deux PlantUML. Un complet et un avec les données souhaitant être gardé. Le choix est libre, il faut juste avoir au moins
	un plantUML complet pour ne pas perdre des infos.
 \item Ce n'est Pas notre probleme. On suppose que le code est valide.
 \item Le client nous a parlé de css pour nous donner un exemple. Mais on utilise ce qu'on veut, le choix est libre.
 \item Pas vraiment. Le choix est assez libre. Il faut par contre que le logiciel soit utilisable à la fac. Il nous a demandé si
	nous allions avoir des contraintes. On lui a dit que pour le moment on en avait pas vu. \textcolor{red}{Mais ça semble important pour lui 
	qu'on y réfléchisse et qu'on le prévienne tôt si on en avait. Je pense notamment à l'execution de JAVAFX sur n'importe quel ordi.}
 \item Le prof ne connait pas cette licence. Il faut qu'on vérifie par nous même. Du moment que le projet est open source, modifiable, portable etc...
\end{enumerate}

\textbf{\underline{Commentaire du client :}}
\begin{itemize}
  \item Attention il faut anticiper le temps de développement et s'y connaitre mieux en JavaFX. Ca semble très important pour la gestion des risques. 
	\textcolor{red}{Il faudrait qu'on se mette rapidement à la technologie qu'on utilisera. (JAVAFX)}
  \item Comment allons faire les test ? Il faudrait qu'on voit ça.
  \item Il faudra des livraisons ponctuels au deuxième semestre (AGILITÉ)
\end{itemize}

\textbf{\underline{Commentaire :}}
\begin{itemize}
  \item Prochaine réunion la semaine prochaine, jour à définir par mail(il ne peut ni jeudi, ni vendredi aprem. Peut etre vendredi midi). 
	Je (Anthony) lui ai promis de lui envoyer rapidement la STB avant la prochaine réunion avec pourquoi pas des diagrammes. 
	(il accepte tous ce qui est exploitable pour le moment, c'est à dire les diagrammes/dessins à la main, sur paint etc...).
  \item M.Nicart nous a conseillé pour le premier TP de gestion de projet de bien résumer notre projet pour le présenter à l'intervenant 
	du TP de gestion de projet pour glaner le plus de conseil possible. Je (Anthony) vais essayer de faire ça ce soir en récupérant ce que vous avez déjà fait.
\end{itemize}

\subsection{Jeudi 09/12/2015}
\underline{Présents: M.Hérauville - Anthony}\\
\textbf{\underline{Sujet: Vérification de la STB}} \\
La STB convient au client mise à part quelques faute de français. Voici quelques informations suplémentaires:
\begin{itemize}
 \item Les diagrammes doivent respecter UML2. Si possible il faudra interdire à un utilisateur de mettre n'importe quelles fleches dans un diagramme.
      Il faudra que l'application le force un minimum à faire des diagrammes correctes. Si possible.
 \item Lire la doc UML2 pourrait être sympa pour être sur de bien la respecter.
 \item A la fin du projet, il faudra lister tous ce q'on a pas reussi à faire respecter de l'UML2 et tous ce qu'on a pu ajouter en plus.
 \item Le client m'a dit qu'il avait discuté avec M.Nicart d'une façon d'implémenter le Reverse. Il nous propose (il ne l'impose pas) à la place 
    d'utiliser un parseur c'est d'utiliser un code recompilé. Je n'ai pas trop compris... Il faudra qu'on en rediscute avec lui.
 \item Il faut vérifier dans la STB que je n'ai pas oublié des fonctionnalités dans les modifications de diagramme par rapport à UML2. Plus on est précis
    mieux c'est. Ca évitera qu'on nous fasse des coups malhonnetes dans le genre ``j'ai essayé de mettre un texte de 1GO dans le titre du diagramme et ça a planté.
    Votre application n'est pas conforme à ce qu'on a signé.''.
 \item Il faut demandé à notre intervenant quels sont les documents que le client doit signer
 \item Le client est pret à signer la STB. Cependant je pense qu'il est utile d'ajouter encore quelques trucs. Quite à retarder un peu la signature.
 Je ne suis pas satisfait de cette STB. Il manque trop de détail par rapport aux exigences d'UML2.
\end{itemize}


\textbf{\underline{Questions :}}
\begin{enumerate}
  \item Nous avons appris que ce projet a déjà été fait par des M1 GIL il y a deux ans. Pourquoi le recommencer ? Quels ont été leurs erreurs ?
  \item Est-on obligé de sauvegarder notre diagramme en plantUML si nous avons une option pour exporter un diagramme en plantUML ?\\
	Personnellement (anthony) je pensais sauvegarder le diagramme en plantUML en y laissant le max d'info et je mettrais avec un dossier
	pour stocker les fichiers de paramètres qui permettre de faire le tri. On aurait deux options pour retirer un élément d'une classe:
	\begin{itemize}
	  \item Cacher un élément: l'élément en question reste enregistré dans le plantUML mais est listé comme étant caché dans les parametres.
		Ceci sert à éviter de devoir regénérer un diagramme à partir d'un code si on supprime un élément qu'on voulait récupérer.
		Mais en vérité cette option est t'elle si utile que ça ?...
	  \item Supprimer un élément: l'élément en question est supprimé du plantUML et est irécupérable.
	  \item Concernant la parti reverse. Quand on génére un diagramme de classe, doit on le faire égalalement pour les package importé ?
	\end{itemize}
\end{enumerate}

\textbf{\underline{Réponses :}} 
\begin{enumerate}
 \item bla
 \item bla
\end{enumerate}

\subsection{Jeudi 27/01/2016}
\underline{Présents: M.Hérauville - Anthonyc- Nabil - Yohann}

\textbf{\underline{Sujet: Nouvelle demande du client}}
Le client a rédéfinie sa demande. Il ne veut plus d'un éditeur de diagramme sous la norme UML2.
Il veut un éditeur de diagramme avec un validateur uml2.
Le modèle est donc simplifié, plus besoin de restriction. On sauvegarde tous ce que peut contenir un diagramme (sous plantuml).

\subsubsection*{Diagramme de classe}

Les 5 entités : Classe, interface, Enumeration et classe abstraite, paquetage.
Les classes et classes abstraites sont la même classe

Une interace peut avoir des attributs ? 
-Oui

Quelles peuvent être les visibilités d'une interface ? De ses attributs et methodes ?
-Le client n'est pas sur il doit vérifier pour les attributs et les méthodes. L'interface peut avoir n'importe quel visibilité. 

Une énumeration peut avoir des attributs/methodes ? 
-Nope

Quelles peuvent être les visibilités d'une enumeration ? 
-Toutes

Quelles peuvent être les visibilités d'une classe ? De ses attributs et methodes ?
-Toutes, toutes, toutes

Restriction pour les paquetages ? On peut les déplacer ? Ca déplace les classes qu'ils contiennent ? Relation entre classe dans paquetage ? (intérieur et extérieur)
-Les relations sont affichées tel quel, les classes se déplacent avec le paquetage puisque leur coordonnée sont calculées relativement au package.
On peut déplacer un package et deux packages ne peuvent posséder le même nom dans un meme package.

Des précisions sur les éléments d'une relation ? Des cardinalités ? Seulement pour un type de relation possible ?
-Il existe plusieurs sortes de relations : les pondérés et les directionnels. Il existe donc 4 sortes de flèches (pour tous les mélanges possible)


\subsubsection*{Diagrammes en general}
Quelles sont vos restriction en fonction des notes ? relations sur les entités ?
-Les notes sont juste une boite avec un possible titre et une chaine de caractères.
Les entitées et les relations contiennent des notes contraintes et des notes basiques (Accrochage de la flèche différent à l'entité, symbole un peu chelou)

\subsubsection*{Notes supplémentaires du modèle de données}
Globalement, on doit éviter le recours au parallèle avec java et plutot travailler à visualiser les diagrammes selon les règles présentes dans son cours. rien de plus, rien de moins.
Les interfaces boucles ne seront pas comprises sauf si le temps le permet.

Techniquement dans une interface, tout le contenu est abstract.

Une classe est abstraite si et seulement si, elle contient au moins une méthode abstraite.

On peut être large sur les contraintes de noms au niveau du modèle de donnée mais pensez à une normalisation pour le plantUML.

Le modèle de données doit pouvoir représenter de l'uml2 meme si plantuml ne le permet pas.
Dans ce sens, il faut prévoir une sauvegarde du modèle différente de plantuml. PlantUML doit etre considéré comme un import/export, non pas comme une sauvegarde.

Il faut pouvoir créer des classe association.

\paragraph{Questions/Réponses}
\subparagraph{Questions}
\begin{enumerate}
 \item Comment voulez vous qu'on manipule les relations entre entité ? Elles se calculent toute seules ? 
	D\&D ? En arrondis ? En ligne droite ? Quelles sont vos exigences précises à ce sujet ?
 \item Rendez-vous la semaine prochaine pour spécifier l'ergonomie ensemble ?
 \item Date précise du rendu du projet ?
\end{enumerate}

\subparagraph{Reponse}
\begin{enumerate}
 \item Trait déplacable et fleche direct. Pas de calcul de trajectoire.
 \item 9h mercredi 3 février
 \item avant les exam
\end{enumerate}


\subsection{Jeudi 03/02/2016}
\underline{Présents: M.Hérauville - Anthony}\newline
\textbf{\underline{Sujet: Spécifier les normes UML2 à implémenter}}\newline
Nouvelle demande du client. Il veut un éditeur de diagramme uml avec un validateur uml2.

\subsection{Jeudi 24/02/2016 à 9h}
\underline{Présents: M.Hérauville - Anthony - Nicolas}\newline
\textbf{\underline{Sujet: Demonstration de la premiere démonstration}}\newline
Le client a apprécié la vue. Cependant il a quelques petites remarques :
\begin{enumerate}
 \item Le zoom doit se faire sur le curseur de souris avec un scroll. Si on zoom depuis le menu, 
	on peut le faire au centre de la vue ou sur le coin gauche.
 \item La création d'entité doit se faire sur le curseur de souris.
 \item La représentation graphicdes entités doit être à peu près la même que plantuml. Par exemple pour les note, les coins doivent 
 être des angle droit sauf celui en haut à droit qui doit être retroussé. Il n'a pas de restriction POUR L'INSTANT concernant le style.
 Pouvoir seulement modifier la couleur de fond de l'entité et de la police lui va actuellement.
\end{enumerate}
Le modèle (au premier abort) lui plait. Petite remarque : On peut définir des type par défaut. 
Par exemple si on crée un attribut ``boujour''. On peut lui associé un type Object ou Property ou autre par défaut.
De même pour la visibilité (public par défaut par exemple). De même pour les méthodes, on peut mettre void par défaut.


\subsection{Vendredi 18/03/2016}
\underline{Présents: M.Hérauville - Anthony - Yohann}\newline
\textbf{\underline{Sujet: Retour du 1er livrable}}\newline
Le client est globalement satisfait de la simplicité et de l'ergonomie du logiciel. On lui a présenter la maquette.
Il veut pouvoir afficher/cacher facilement les barre d'outil. Il approuve le ``linkage'' des entités (sans copie, la suppression d'une methode d'une classe C dans un diagramme 
la supprime de la banque de donnée et donc de tous les diagramme qui contenait cette entité.)
Bug lors de la création d'une entité dans le diagramme sur la position lorsque c'est zommé.``
Problème de police, on a pas les droits sur une police window.
Le client sera indisponible la deuxieme semaine des vacances. On va donc lui envoyer un livrable dès la premiere semaine des vacances en lui demandant 
un retour.

\section{Réunions équipe}
\subsection{Jeudi 15/10/2015}
\underline{Présents: Toute l'équipe}
\\
\begin{itemize}
  \item Rencontre de l'équipe dans un lieu conviviale. Tous le monde s'est présenté et a dressé une petite fiche renseignant 
    sur ses qualités et ses défauts en projet informatique.
  \item Désignation d'un chef de projet: Florian Inchingolo. 
\end{itemize}

\subsection{Mardi 20/10/2015}
\underline{Présents: Toute l'équipe}
\\
\begin{itemize}
  \item Nous avons commencé à réfléchir/débattre à l'architecture que notre projet pourrait avoir. 
  \item Désignation des chargés de communication avec le client : Anthony Godin, Nabil Belkhous.
\end{itemize}
 
\subsection{Mardi 03/11/2015}
\underline{Présents: Anthony - Florian - Yohann}
\\
\begin{itemize}
  \item Récap' et construction de la structure du projet. Plusieurs idées et questions sont abordées. La prochaine réunion avec le client
	nous permettra de prendre des décisions importantes pour la structure.
\end{itemize}


\subsection{Mercredi 04/11/2015}
\underline{Présents: Anthony - Florian - Nabil - Nicolas - Stephen - Yohann} 
\\
\begin{itemize}
  \item Récap' de la dernière réunion. Tous le monde semble d'accord avec la structure imaginée.
  \item Répartition du travail. Il y aura 3 parties: \begin{enumerate}
                                                      \item IHM
                                                      \item Ensemble d'objet (la partie qui modélise les info virtuellement + les options à gérer)
                                                      \item Acteur (parseurs)
                                                     \end{enumerate}
\end{itemize}
A la prochaine réunion (le lendemain) il faudra commencer à lister préciséments les options de l'application.

\subsection{Jeudi 05/11/2015}
\underline{Présents: Anthony - Florian - Nabil - Saad - Stephen - Yohann} 
Début de STB et des diagramme de cas d'utilisation.

\subsection{Vendredi 06/11/2015}
\underline{Présents: Toute l'équipe} 
Pour le TP2 de Gestion de projet:\\
\begin{itemize}
 \item \textcolor{red}{\underline{Finaliser la STB}}
 \item Revue de la STB avec le client
 \item Réunion Hebdo avec le client
 \item Lister les difficultés du projet
\end{itemize}

Template d'un compte rendu de réunion:
\begin{itemize}
 \item Date/ Personne/ Sujet
 \item Contexte\\
 \item Action\\
 \item Sujet abordé
\end{itemize}


\subsection{Jeudi 11/11/2015}
\underline{Présents: Anthony - Florian - Nabil - Nicolas - Saad - Yohann} \newline
Sujet: Travail pour la STB 
\begin{itemize}
 \item Construction de la structure de la STB
 \item Reconstruction des diagrammes des cas
 \item Commencement des cas d'utilisation
 \item Premier test d'interface graphique
\end{itemize}

\subsection{Jeudi 29/01/2016}
\underline{Présents: Anthony - Florian - Nicolas - Saad - Yohann - Stephen}  \newline
\textbf{Sujet} : Présentation de notre PDD, plannification de sprint et changement de chef de projet. 
\newline

Nous avons fait une mélée. Globalement on a tous installé notre environement de travail sur notre pc cette semaine.\newline
Florian démissionne de son statut de chf de projet. Il ne se sent plus apte à tenir ce rôle à cause de problèmes personnels.
Une éléction d'un nouveau chef de projet est prévu sur Slack.\newline
Le PDD est refait. On y intégre Scrum. Nous l'avons présenté à l'équipe avec le diagramme de Gantt.\newline
La préparation du sprint est faite. Nous avons mis en place un BackLog sur Trello.\newline
L'itération 1 débute offciellement.\newline
Pour la premiere semaine :
\begin{itemize}
 \item Vue : Stephen(responsable), Saad et Anthony
 \item Modèle de diagramme : Yohann (responsable), Nabil et Nico
 \item Parseur : Florian (responsable)2524
\end{itemize}

\subsection{Jeudi 29/01/2016}
\underline{Présents: Toute l'équipe}  \newline
\textbf{Sujet} : Petite mélée. Rappel des taches de la semaine de chacun.
\newline
A faire :
\begin{itemize}
 \item Elire un nouveau chef de projet
 \item Reparler de l'ergonomie rapidement
 \item Rappeler l'attribution des taches
 \item Faire un point sur le backlog. Les trois responsables doivent s'assurer que les 
 dates sont bien fixée pour chacunes des taches. 
 Que ça correspondent bien avec le Gantt. Il faut à tous pris éviter le retard...
\end{itemize}
Niveau ergonomie : Quand on ajoute une entité, une pop-up s'affiche et demande son nom et ses d'elements si l'utilisateur le veut.

\subsection{Jeudi 04/02/2016}
\underline{Présents: Anthony, Nabil, Nicolas, Stephen, Yohann}  \newline
\textbf{Sujet} : Mélée. Rappel des taches de la semaine de chacun. Compte rendu de réunion client
\newline
Rappel: On code en anglais. Les nom de toutes les fonctions doivent être en ANGLAIS. \newline
Explication du nouveau modèle à l'équipe et de la nouvelle vue. Nouvelle planification des équipes jusqu'au 12 février :
\begin{itemize}
 \item Vue : Stephen, Anthony, Saad et Nico
 \item Modèle : Yohann et Nabil
 \item Parseur : Florian
\end{itemize}
Prochaine réunion prévu : Lundi 8 février après midi chez Yohann ou chez Anthony. Mélée + développement en groupe : 4h.\newline
Un nouveau PDD avec un nouveau planning est prévu pour le 8 février.
\subsubsection*{Nouveau modèle}
Le nouveau modèle ne fait plus de restriction. Il peut contenir tous ce qu'on peut interpréter en uml. Attention tout de même,
un diagramme de classe ne peut pas contenir un cas d'utilisation.
Le modèle d'un diagramme de classe contient :
\begin{itemize}
 \item Des entités (classe, interface etc...) avec un id.
 \item Des relations qui contiennent un id ainsi que l'id des entités source et destination de la relation.
\end{itemize}
\subsubsection*{Nouvelle vue}
Il faudra ajouter dans la vue une nouvelle boite qui permet d'afficher toutes les entités et relation que contient un projet avec la possibilité de les éditer.
Evidemment de manière ergonomique. Le chargé clientèle travaillera bientot ça avec le client.
\subsubsection*{Nouvelle responsabilité}
Saad devient responsable du menu. Il gerera sa partie visuelle ainsi que le controleur.

\subsection{Lundi 21/02/2016}
\underline{Présents: Toute l'équipe}  \newline
\textbf{Sujet} : Mélée. Gros point sur le travail des vacances, sur le travail sur le gantt.
\newline
Attribution des taches, préparation de la réunion/présentation avec le client. \newline
Présentation des structures logiciels des 3 parties et tous le monde vérifie que les parties sont bien compatibles. \newline
Discution avec le modèle de diagramme des property !

\end{document}