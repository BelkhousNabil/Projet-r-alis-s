\documentclass[hidelinks, a4paper,11pt,twoside,final]{article}

\usepackage[english,francais]{babel}
\usepackage[utf8]{inputenc}
\usepackage{geometry}
\usepackage[T1]{fontenc}
\usepackage[pdftex]{graphicx}
\usepackage{adjustbox}
\usepackage{color}
\usepackage{setspace}
\usepackage{hyperref}
\usepackage[french]{varioref}
\usepackage{comment}
\usepackage{dirtree}


\usepackage{fancyhdr}
\pagestyle{fancy}

\renewcommand{\headrulewidth}{1pt}
\fancyhead[L]{}
\fancyhead[C]{\textbf{UML Reverse}}
\fancyhead[R]{\includegraphics[width=2cm]{img/universite-rouen.jpg}}

%Opening
\title{\bfseries Cahier des recettes\\Projet UML Reverse}
\geometry{hmargin=2.5cm,vmargin=3cm}
\begin{document}
\maketitle
\begin{center}
\begin{tabular}{ll}
  Version~: & 0.1\\[.5em]
  Date~: & \date{\today}\\[.5em]
  Rédigé par~: & Nabil \textsc{Belkhous}\\
               & Anthony \textsc{Godin}\\[.5em]
  Relu par~:   & Anthony \textsc{Godin}\\
               & Florian \textsc{Inchingolo}\\
               & Nicolas \textsc{Meniel}\\[.5em]
\end{tabular}
\end{center}

%Table of contents
\newpage
\tableofcontents

%Contents
\newpage
\section{Introduction}
  Ce document a le but de définir les tests nous permettant de prouver que le projet en cours d’élaboration respecte bien le contrat
  en accord avec le client.\\
  Chaque cas d’utilisation a un test défini, ainsi que chaque exigence fonctionnelle.

\section{Cas d'utilisation et exigences fonctionnelles}
  Voir le document STB.

\section{Terminologie et sigles utilisés}
  Se référer au document Terminologie.
  \begin{description}
    \item[Test de modification d’élément/entité~:]~\\ Ici on veut dire tester toutes les modifications possibles du contenu de l’élément/entité défini dans la STB.
    Ajout, suppression, cachement, affichage du contenu.
    \item[(Action) de toutes les manières possibles~:]~\\ Nous n’avons pas encore défini toutes les manières de réaliser une action graphiquement.
    Actuellement des boutons sont prévus, mais le client pourra être plus précis au moment où nous aurons une version à lui présenter.
    Il y aura un manuel d’utilisation qui listera toutes les manières possibles d’exécuter une action. Il faudra se référer à ce document pour tester toutes
    les manières possibles d’exécuter une action.
  \end{description}


\section{Environnement de test}
    Le projet est produit avec Maven. Maven nous servira de gestionnaire de test. Chaque fichier de classe a un fichier de test qui lui est associé.
    Le code source sera dans le dossier main et les tests dans un dossier test. La compilation d’une classe relancera tous les tests qui lui sont associés.
    Bien sûr, une fois qu’une classe est compilée, elle le sera à nouveau seulement si elle est modifiée, ce qui forcera à refaire tous les tests.
    JUnit est intégré dans Maven. Voici l’arboresence des fichiers.
    \dirtree{%
      .1 src.
      .2 main.
      .3 java.
      .4 com.
      .5 monprojet.
      .6 Class.java.
      .2 test.
      .3 java.
      .4 com.
      .5 monprojet.
      .6 ClassTest.java.
    }

\section{Responsabilités et stratégie}
  \begin{itemize}
   \item \textbf{Responsable des tests}~: Nabil \textsc{Belkhous}
  \end{itemize}
  Le responsable des tests est la personne chargée de vérifier que tous les tests annoncés dans ce document sont validés.\\
  Les tests seront écrits par l’ensemble de l’équipe. Chaque membre écrit les tests du code qu’il produit. Les développeurs ne coderont qu’une classe à la fois.
  Toute classe devra être testée et approuvée avant d’être définie comme terminée. Le développeur pourra ensuite (et seulement ensuite) passer
  au développement de la classe suivante.\\
  Les tests seront effectués à l’aide de JUnit et Maven de la façon décrite dans la section «~Environnement de test~».

\section{Tests fonctionnels}
  On ne notera pas les contraintes de bon sens. Il est évident qu’on a besoin d’un minimum de mémoire quand on enregistre un fichier. On notera aussi qu’on ne peut pas nommer deux
  fichiers avec le même nom s’ils sont dans le même dossier.

  \subsection*{Créer un projet}
    \begin{tabular}{|r|p{5cm}|p{5cm}|}\hline
	{Identifiant du test~:} & \multicolumn{2}{|p{10cm}|}{TEST\_1} \\\hline
	{Identifiant de l’exigence~:} & \multicolumn{2}{|p{10cm}|}{IHM\_10} \\\hline
        {Objet~:} & \multicolumn{2}{|p{10cm}|}{Créer un projet} \\\hline
        {Contraintes~:} & \multicolumn{2}{|p{10cm}|}{} \\\hline
        {Action~:} & \multicolumn{2}{|p{10cm}|}{Créer un projet de toutes les manières possibles.} \\\hline
        {Résultat attendu~:} & \multicolumn{2}{|p{10cm}|}{L’application demande à l’utilisateur de donner le chemin où
                              il veut enregistrer son projet sur son ordinateur.
                              Une fois donné, un dossier au chemin donné par l’utilisateur sera créé.
                              } \\\hline
        {État~:} & {\textcolor{green}{\textbf{Fait}}} & {Le~: 03/05/2016 } \\\hline
    \end{tabular}
    \\
    \newline

    \subsection*{Ouvrir un projet}
    \begin{tabular}{|r|p{5cm}|p{5cm}|}\hline
	{Identifiant~:} & \multicolumn{2}{|p{10cm}|}{TEST\_2} \\\hline
	{Identifiant de l’exigence~:} & \multicolumn{2}{|p{10cm}|}{IHM\_20} \\\hline
        {Objet~:} & \multicolumn{2}{|p{10cm}|}{Ouvrir un projet} \\\hline
        {Contraintes~:} & \multicolumn{2}{|p{10cm}|}{Il faut que le projet existe.} \\\hline
        {Action~:} & \multicolumn{2}{|p{10cm}|}{Ouvrir un projet de toutes les manières possibles.} \\\hline
        {Résultat attendu~:} & \multicolumn{2}{|p{10cm}|}{L’application demande à l’utilisateur de lui donner le chemin du projet.
                             Les dossiers dans les projets sont bien lus et affichés par l’application, ainsi que tous les diagrammes présents
                             dans les fichiers src avec leurs fichiers de paramètres.} \\\hline
        {État~:} & {\textcolor{green}{\textbf{Fait}}} & {Le~: 03/05/2016 } \\\hline
    \end{tabular}
    \\
    \newline

     \subsection*{Supprimer un projet}
    \begin{tabular}{|r|p{5cm}|p{5cm}|}\hline
    {Identifiant~:} & \multicolumn{2}{|p{10cm}|}{TEST\_3} \\\hline
    {Identifiant de l’exigence~:} & \multicolumn{2}{|p{10cm}|}{IHM\_20} \\\hline
        {Objet~:} & \multicolumn{2}{|p{10cm}|}{Supprimer un projet} \\\hline
        {Contraintes~:} & \multicolumn{2}{|p{10cm}|}{Le projet doit exister} \\\hline
        {Action~:} & \multicolumn{2}{|p{10cm}|}{Supprimer un projet de toutes les manières possibles.} \\\hline
        {Résultat attendu~:} & \multicolumn{2}{|p{10cm}|}{L’application demande à l’utilisateur de sélectionner le projet.
                              Le dossier du projet selectionné est supprimé avec tous ses dossiers fils} \\\hline
        {État~:} & {\textcolor{green}{\textbf{Fait}}} & {Le~: 03/05/2016 } \\\hline
    \end{tabular}
    \\
    \newline

    \subsection*{Sauvegarder un projet}
    \begin{tabular}{|r|p{5cm}|p{5cm}|}\hline
    {Identifiant~:} & \multicolumn{2}{|p{10cm}|}{TEST\_4} \\\hline
    {Identifiant de l’exigence~:} & \multicolumn{2}{|p{10cm}|}{IHM\_20} \\\hline
        {Objet~:} & \multicolumn{2}{|p{10cm}|}{Sauvegarder un projet} \\\hline
        {Contraintes~:} & \multicolumn{2}{|p{10cm}|}{Il faut que le projet existe.} \\\hline
        {Action~:} & \multicolumn{2}{|p{10cm}|}{Sauvegarder un projet de toutes les manières possibles} \\\hline
        {Résultat attendu~:} & \multicolumn{2}{|p{10cm}|}{L’application demande à l’utilisateur de lui spécifier quel projet il veut sauvegarder si ce n’est pas déjà fait.
                            Les diagrammes (src et paramètres) du projet ont tous été sauvegardés.} \\\hline
    {Test supplémentaire~:} & \multicolumn{2}{|p{10cm}|}{Si l’utilisateur charge un projet juste après l’avoir sauvegardé, il devrait
                                pouvoir visualiser les mêmes diagrammes qu’au moment de la sauvegarde.} \\\hline
        {État~:} & {\textcolor{green}{\textbf{Fait}}} & {Le~: 03/05/2016 } \\\hline
    \end{tabular}
    \\
    \newline

    \subsection*{Créer un diagramme}
    \begin{tabular}{|r|p{5cm}|p{5cm}|}\hline
    {Identifiant~:} & \multicolumn{2}{|p{10cm}|}{TEST\_5} \\\hline
    {Identifiant de l’exigence~:} & \multicolumn{2}{|p{10cm}|}{IHM\_30 - DIA\_10 - DIA\_30 - DIA\_40 - DIA\_50} \\\hline
        {Objet~:} & \multicolumn{2}{|p{10cm}|}{Créer un diagramme} \\\hline
        {Contraintes~:} & \multicolumn{2}{|p{10cm}|}{} \\\hline
        {Action~:} & \multicolumn{2}{|p{10cm}|}{Créer un diagramme de toutes les manières possibles} \\\hline
        {Résultat attendu~:} & \multicolumn{2}{|p{10cm}|}{L’application demande à l’utilisateur de sélectionner dans quel projet il veut enregistrer le diagramme ou
                             en créer un.
                             Elle lui demande également le type du diagramme (cas d’utilisation, séquence ou classe) ainsi qu’un nom.
                                     Le nom ne pourra pas être le même qu’un autre diagramme du même type du même projet. Si c’est le cas,
                                     l’application le signalera et proposera à l’utilisateur d’écraser le fichier correspondant.
                             Initialise les objets internes à l’application du type de diagramme en question.
                             Le diagramme est affiché (vide) sur l’application (partie graphique).
                             Deux fichiers du nom du diagramme sont créés dans les dossiers src et parameters
                             dans le dossier du type du diagramme du projet séléctionné.} \\\hline
        {État~:} & {\textcolor{green}{\textbf{Fait}}} & {Le~: 03/05/2016 } \\\hline
    \end{tabular}
    \\
    \newline

    \subsection*{Sauver un diagramme}
    \begin{tabular}{|r|p{5cm}|p{5cm}|}\hline
    {Identifiant~:} & \multicolumn{2}{|p{10cm}|}{TEST\_6} \\\hline
    {Identifiant de l’exigence~:} & \multicolumn{2}{|p{10cm}|}{IHM\_30} \\\hline
        {Objet~:} & \multicolumn{2}{|p{10cm}|}{Sauver un diagramme} \\\hline
        {Contraintes~:} & \multicolumn{2}{|p{10cm}|}{Il faut que le diagramme existe} \\\hline
        {Action~:} & \multicolumn{2}{|p{10cm}|}{Sauver un diagramme de toutes les manières possibles} \\\hline
        {Résultat attendu~:} & \multicolumn{2}{|p{10cm}|}{Les modifications apportés sur le digramme depuis sa dernière sauvegarde ont été enregistrés
        dans les fichiers lui correspondant} \\\hline
    {Test supplémentaire~:} & \multicolumn{2}{|p{10cm}|}{Si l’utilisateur charge un diagramme juste après l’avoir sauvegardé, il devrait
                                pouvoir visualiser le même diagramme qu’au moment de la sauvegarde.} \\\hline
        {État~:} & {\textcolor{green}{\textbf{Fait}}} & {Le~: 03/05/2016 } \\\hline
    \end{tabular}
    \\
    \newline

    \subsection*{Supprimer un diagramme}
    \begin{tabular}{|r|p{5cm}|p{5cm}|}\hline
    {Identifiant~:} & \multicolumn{2}{|p{10cm}|}{TEST\_7} \\\hline
    {Identifiant de l’exigence~:} & \multicolumn{2}{|p{10cm}|}{IHM\_30} \\\hline
        {Objet~:} & \multicolumn{2}{|p{10cm}|}{Supprimer un diagramme} \\\hline
        {Contraintes~:} & \multicolumn{2}{|p{10cm}|}{Il faut que le diagramme existe} \\\hline
        {Action~:} & \multicolumn{2}{|p{10cm}|}{Supprimer un diagramme de toutes les manières possibles} \\\hline
        {Résultat attendu~:} & \multicolumn{2}{|p{10cm}|}{Les fichiers correspondants seront supprimés (dans src et parameters).} \\\hline
        {État~:} & {\textcolor{green}{\textbf{Fait}}} & {Le~: 03/05/2016 } \\\hline
    \end{tabular}
    \\
    \newline

    \subsection*{Ouvrir un diagramme}
    \begin{tabular}{|r|p{5cm}|p{5cm}|}\hline
    {Identifiant~:} & \multicolumn{2}{|p{10cm}|}{TEST\_8} \\\hline
    {Identifiant de l’exigence~:} & \multicolumn{2}{|p{10cm}|}{IHM\_30} \\\hline
        {Objet~:} & \multicolumn{2}{|p{10cm}|}{Ouvrir un diagramme} \\\hline
        {Contraintes~:} & \multicolumn{2}{|p{10cm}|}{Le diagramme doit exister} \\\hline
        {Action~:} & \multicolumn{2}{|p{10cm}|}{Ouvrir un diagramme de toutes les manières possibles} \\\hline
        {Résultat attendu~:} & \multicolumn{2}{|p{10cm}|}{L’application demande à l’utilisateur de sélectionner un diagramme.
                              Le diagramme est affiché sur l’application (partie graphique)} \\\hline
        {État~:} & {\textcolor{green}{\textbf{Fait}}} & {Le~: 03/05/2016 } \\\hline
    \end{tabular}
    \\
    \newline
    
     \subsection*{Charger un fichier PlantUML}
    \begin{tabular}{|r|p{5cm}|p{5cm}|}\hline
    {Identifiant~:} & \multicolumn{2}{|p{10cm}|}{TEST\_8} \\\hline
    {Identifiant de l’exigence~:} & \multicolumn{2}{|p{10cm}|}{IHM\_30} \\\hline
        {Objet~:} & \multicolumn{2}{|p{10cm}|}{Charger un fichier PlantUML} \\\hline
        {Action~:} & \multicolumn{2}{|p{10cm}|}{Charger un ficier PlantUML de toutes les façons possibles} \\\hline
        {Résultat attendu~:} & \multicolumn{2}{|p{10cm}|}{Un diagramme est chargé en mémoire et affiché graphiquement tel qu’il est spécifié dans le fichier PlantUML
        Un fichier de style lui a été affecté.} \\\hline
        {État~:} & {\textcolor{green}{\textbf{Fait}}} & {Le~: 03/05/2016 } \\\hline
    \end{tabular}
    \\
    \newline
    
    
    \subsection*{Zoomer/Dézoomer un diagramme}
    \begin{tabular}{|r|p{5cm}|p{5cm}|}\hline
    {Identifiant~:} & \multicolumn{2}{|p{10cm}|}{TEST\_9} \\\hline
    {Identifiant de l’exigence~:} & \multicolumn{2}{|p{10cm}|}{IHM\_40} \\\hline
        {Objet~:} & \multicolumn{2}{|p{10cm}|}{Zoomer un diagramme} \\\hline
        {Contraintes~:} & \multicolumn{2}{|p{10cm}|}{Le diagramme doit être ouvert} \\\hline
        {Action~:} & \multicolumn{2}{|p{10cm}|}{Zoomer/Dézoomer un diagramme de toutes les manières possibles} \\\hline
        {Résultat attendu~:} & \multicolumn{2}{|p{10cm}|}{On voit visuellement que le zoom/dézoom est appliqué.} \\\hline
        {État~:} & {\textcolor{green}{\textbf{Fait}}} & {Le~: 03/05/2016 } \\\hline
    \end{tabular}
    \\
    \newline
    
     \subsection*{Exporter un diagramme en PlantUML}
    \begin{tabular}{|r|p{5cm}|p{5cm}|}\hline
    {Identifiant~:} & \multicolumn{2}{|p{10cm}|}{TEST\_12} \\\hline
    {Identifiant de l’exigence~:} & \multicolumn{2}{|p{10cm}|}{IHM\_70 - EXF\_50} \\\hline
        {Objet~:} & \multicolumn{2}{|p{10cm}|}{Exporter un diagramme en PlantUML} \\\hline
        {Contraintes~:} & \multicolumn{2}{|p{10cm}|}{Il faut que le diagramme existe et soit ouvert} \\\hline
        {Action~:} & \multicolumn{2}{|p{10cm}|}{Exporter un diagramme en PlantUML de toutes les manières possibles} \\\hline
        {Résultat attendu~:} & \multicolumn{2}{|p{10cm}|}{L’application demande à l’utilisateur où il veut enregistrer le fichier.
                             Un fichier PlantUML représentant le diagramme est créé où l’utilisateur l’a demandé.} \\\hline
        {État~:} & {\textcolor{green}{\textbf{Fait}}} & {Le~: 03/05/2016 } \\\hline
    \end{tabular}
    \\
    \newline


    \subsection*{Exporter un diagramme en image}
    \begin{tabular}{|r|p{5cm}|p{5cm}|}\hline
    {Identifiant~:} & \multicolumn{2}{|p{10cm}|}{TEST\_13} \\\hline
    {Identifiant de l’exigence~:} & \multicolumn{2}{|p{10cm}|}{IHM\_80} \\\hline
        {Objet~:} & \multicolumn{2}{|p{10cm}|}{Exporter un diagramme en image} \\\hline
        {Contraintes~:} & \multicolumn{2}{|p{10cm}|}{Il faut que le diagramme existe et soit ouvert} \\\hline
        {Action~:} & \multicolumn{2}{|p{10cm}|}{Exporter un diagramme en image de toute les manières possibles} \\\hline
        {Résultat attendu~:} & \multicolumn{2}{|p{10cm}|}{L’application demande à l’utilisateur où il veut enregistrer l’image.
                              Puis un fichier image est créé où l’utilisateur l’a demandé et représentant
                              visuellement la même chose que le diagramme affiché par l’application} \\\hline
        {État~:} & {\textcolor{green}{\textbf{Fait}}} & {Le~: 03/05/2016 } \\\hline
    \end{tabular}
    \\
    \newline

    \subsection*{Imprimer un diagramme}
    \begin{tabular}{|r|p{5cm}|p{5cm}|}\hline
    {Identifiant~:} & \multicolumn{2}{|p{10cm}|}{TEST\_14} \\\hline
    {Identifiant de l’exigence~:} & \multicolumn{2}{|p{10cm}|}{IHM\_90} \\\hline
        {Objet~:} & \multicolumn{2}{|p{10cm}|}{Imprimer un diagramme} \\\hline
        {Contraintes~:} & \multicolumn{2}{|p{10cm}|}{Il faut que le diagramme existe et soit ouvert} \\\hline
        {Action~:} & \multicolumn{2}{|p{10cm}|}{Imprimer un diagramme de toutes les manières possibles} \\\hline
        {Résultat attendu~:} & \multicolumn{2}{|p{10cm}|}{L’impression s’exécute correctement} \\\hline
        {État~:} & {\textcolor{green}{\textbf{Fait}}} & {Le~: 03/05/2016 } \\\hline
    \end{tabular}
    \\
    \newline


    \subsection*{Exporter un diagramme UML en PlantUML}
    \begin{tabular}{|r|p{5cm}|p{5cm}|}\hline
    {Identifiant~:} & \multicolumn{2}{|p{10cm}|}{TEST\_15} \\\hline
        {Objet~:} & \multicolumn{2}{|p{10cm}|}{Exporter un diagramme UML en PlantUML} \\\hline
        {Contraintes~:} & \multicolumn{2}{|p{10cm}|}{Il faut que le diagramme existe et soit ouvert} \\\hline
        {Action~:} & \multicolumn{2}{|p{10cm}|}{Exporter un diagramme UML en PlantUML de toutes les manières possibles} \\\hline
        {Résultat attendu~:} & \multicolumn{2}{|p{10cm}|}{L’application demande à l’utilisateur où il veut enregistrer le dossier.
                             Un dossier représentant le diagramme est créé où l’utilisateur l’a demandé.} \\\hline
        {Test supplémentaire~:} & \multicolumn{2}{|p{10cm}|}{Importer le dossier exporté visuellement le même diagramme.
							    Au moment de l’export d’un diagramme, sauvegarder le diagramme (celui qui est modélisé).
							    Recharger le diagramme et comparer la première modélisation à la deuxième. } \\\hline
        {État~:} & {\textcolor{green}{\textbf{Fait}}} & {Le~: 03/05/2016 } \\\hline
    \end{tabular}
    \\
    \newline
    
    \subsection*{Importer un PlantUML en diagramme UML}
    \begin{tabular}{|r|p{5cm}|p{5cm}|}\hline
    {Identifiant~:} & \multicolumn{2}{|p{10cm}|}{TEST\_16} \\\hline
    {Identifiant de l’exigence~:} & \multicolumn{2}{|p{10cm}|}{IHM\_100} \\\hline
        {Objet~:} & \multicolumn{2}{|p{10cm}|}{Importer un PlantUML en diagramme UML} \\\hline
        {Action~:} & \multicolumn{2}{|p{10cm}|}{Importer un PlantUML en diagramme UML de toutes les manières possibles} \\\hline
        {Résultat attendu~:} & \multicolumn{2}{|p{10cm}|}{L’application charge et affiche le diagramme en question} \\\hline
        {Test supplémentaire~:} & \multicolumn{2}{|p{10cm}|}{Importer le dossier exporté visuellement le même diagramme.
							    Au moment de l’export d’un diagramme, sauvegarder le diagramme (celui qui est modélisé).
							    Recharger le diagramme et comparer la première modélisation à la deuxième. } \\\hline
        {État~:} & {\textcolor{green}{\textbf{Fait}}} & {Le~: 03/05/2016 } \\\hline
    \end{tabular}
    \\
    \newline

    
    \subsection*{Reverse}
    \begin{tabular}{|r|p{5cm}|p{5cm}|}\hline
    {Identifiant~:} & \multicolumn{2}{|p{10cm}|}{TEST\_17} \\\hline
    {Identifiant de l’exigence~:} & \multicolumn{2}{|p{10cm}|}{DIA\_60 - DIA\_70 - DIA\_80 - EXF\_30} \\\hline
        {Objet~:} & \multicolumn{2}{|p{10cm}|}{Reverse} \\\hline
        {Action~:} & \multicolumn{2}{|p{10cm}|}{Générer un diagramme en Reverse de toutes les manières possibles sur un fichier écrit en Java 1.7} \\\hline
        {Résultat attendu~:} & \multicolumn{2}{|p{10cm}|}{Toutes les classes du code source sont bien représentées dans le diagramme} \\\hline
        {État~:} & {\textcolor{green}{\textbf{Fait}}} & {Le~: 03/05/2016 } \\\hline
    \end{tabular}
    \\
    \newpage

    \subsection*{Modification d’un diagramme}
    Ici on testera les fonctionnalités de la modification de diagramme visuellement.
    Les cases correspondant aux fonctionnalités testées et approuvées doivent être remplies par \textcolor{green}{\textbf{Fait}}.\\
    \\
    Tests d’ajout, de suppression, d’affichage, de cachement et de modification d’éléments~:\\
    \begin{tabular}{|c|c|c|c|c|}\hline
    {Identifiant~:} & \multicolumn{4}{|l|}{TEST\_18} \\\hline
    {Identifiant de l’exigence~:} & \multicolumn{4}{|p{10cm}|}{MOD\_10 - MOD\_20 - MOD\_30 - MOD\_60 - MOD\_80 - MOD\_90 - MOD\_100} \\\hline
    \multicolumn{5}{|c|}{\textbf{Diagramme de cas d’utilisation}} \\\hline
    {\textbf{Éléments}} & {Ajouter} & {Supprimer} & {Afficher/cacher} & {Modification} \\\hline
    {Relation de généralisation} & {\textcolor{green}{\textbf{Fait}}} & {\textcolor{green}{\textbf{Fait}}} & {\textcolor{green}{\textbf{Fait}}} & {\textcolor{green}{\textbf{Fait}}}\\\hline
    {Relation d’extension} & {\textcolor{green}{\textbf{Fait}}} & {\textcolor{green}{\textbf{Fait}}} & {\textcolor{green}{\textbf{Fait}}} & {\textcolor{green}{\textbf{Fait}}}\\\hline
    {Relation d’inclusion} & {\textcolor{green}{\textbf{Fait}}} & {\textcolor{green}{\textbf{Fait}}} & {\textcolor{green}{\textbf{Fait}}} & {\textcolor{green}{\textbf{Fait}}}\\\hline
    {Relation d’association} & {\textcolor{green}{\textbf{Fait}}} & {\textcolor{green}{\textbf{Fait}}} & {\textcolor{green}{\textbf{Fait}}} & {\textcolor{green}{\textbf{Fait}}}\\\hline
    {Note} & {\textcolor{green}{\textbf{Fait}}} & {\textcolor{green}{\textbf{Fait}}} & {\textcolor{green}{\textbf{Fait}}} & {\textcolor{green}{\textbf{Fait}}}\\\hline
    {Frontière} & {\textcolor{yellow}{\textbf{Annulé (Optionnel)}}} & {\textcolor{yellow}{\textbf{Annulé (Optionnel)}}} & {\textcolor{yellow}{\textbf{Annulé (Optionnel)}}} & {\textcolor{yellow}{\textbf{Annulé (Optionnel)}}}\\\hline
    {Acteur} & {\textcolor{green}{\textbf{Fait}}} & {\textcolor{green}{\textbf{Fait}}} & {\textcolor{green}{\textbf{Fait}}} & {\textcolor{green}{\textbf{Fait}}}\\\hline
    {Cas d’utilisation} & {\textcolor{green}{\textbf{Fait}}} & {\textcolor{green}{\textbf{Fait}}} & {\textcolor{green}{\textbf{Fait}}} & {\textcolor{green}{\textbf{Fait}}}\\\hline
    \multicolumn{5}{|c|}{\textbf{Diagramme de classe}} \\\hline
    {\textbf{Éléments}} & {Ajouter} & {Supprimer} & {Afficher/cacher} & {Modification}\\\hline
    {Relation d’implémentation} & {\textcolor{green}{\textbf{Fait}}} & {\textcolor{green}{\textbf{Fait}}} & {\textcolor{green}{\textbf{Fait}}} & {\textcolor{green}{\textbf{Fait}}}\\\hline
    {Relation d’héritage} & {\textcolor{green}{\textbf{Fait}}} & {\textcolor{green}{\textbf{Fait}}} & {\textcolor{green}{\textbf{Fait}}} & {\textcolor{green}{\textbf{Fait}}}\\\hline
    {Relation de composition} & {\textcolor{green}{\textbf{Fait}}} & {\textcolor{green}{\textbf{Fait}}} & {\textcolor{green}{\textbf{Fait}}} & {\textcolor{green}{\textbf{Fait}}}\\\hline
    {Relation d’agrégation} & {\textcolor{green}{\textbf{Fait}}} & {\textcolor{green}{\textbf{Fait}}} & {\textcolor{green}{\textbf{Fait}}} & {\textcolor{green}{\textbf{Fait}}}\\\hline
    {Relation d’association} & {\textcolor{green}{\textbf{Fait}}} & {\textcolor{green}{\textbf{Fait}}} & {\textcolor{green}{\textbf{Fait}}} & {\textcolor{green}{\textbf{Fait}}}\\\hline
    {Relation de dépendance} & {\textcolor{green}{\textbf{Fait}}} & {\textcolor{green}{\textbf{Fait}}} & {\textcolor{green}{\textbf{Fait}}} & {\textcolor{green}{\textbf{Fait}}}\\\hline
    {Énumération} & {\textcolor{green}{\textbf{Fait}}} & {\textcolor{green}{\textbf{Fait}}} & {\textcolor{green}{\textbf{Fait}}} & {\textcolor{green}{\textbf{Fait}}}\\\hline
    {Classe abstraite} & {\textcolor{green}{\textbf{Fait}}} & {\textcolor{green}{\textbf{Fait}}} & {\textcolor{green}{\textbf{Fait}}} & {\textcolor{green}{\textbf{Fait}}}\\\hline
    {Interface} & {\textcolor{green}{\textbf{Fait}}} & {\textcolor{green}{\textbf{Fait}}} & {\textcolor{green}{\textbf{Fait}}} & {\textcolor{green}{\textbf{Fait}}}\\\hline
    {Classe} & {\textcolor{green}{\textbf{Fait}}} & {\textcolor{green}{\textbf{Fait}}} & {\textcolor{green}{\textbf{Fait}}} & {\textcolor{green}{\textbf{Fait}}}\\\hline
    {Note} & {\textcolor{green}{\textbf{Fait}}} & {\textcolor{green}{\textbf{Fait}}} & {\textcolor{green}{\textbf{Fait}}} & {\textcolor{green}{\textbf{Fait}}}\\\hline
    {Paquetage} & {\textcolor{yellow}{\textbf{Annulé (Optionnel)}}} & {\textcolor{yellow}{\textbf{Annulé (Optionnel)}}} & {\textcolor{yellow}{\textbf{Annulé (Optionnel)}}} & {\textcolor{yellow}{\textbf{Annulé (Optionnel)}}}\\\hline
    \multicolumn{5}{|c|}{\textbf{Diagramme de séquence}}\\\hline
    {\textbf{Éléments}} & {Ajouter} & {Supprimer} & {Afficher/cacher} & {Modification} \\\hline
    {Message} & {\textcolor{yellow}{\textbf{Annulé (Optionnel)}}} & {\textcolor{yellow}{\textbf{Annulé (Optionnel)}}} & {\textcolor{yellow}{\textbf{Annulé (Optionnel)}}} & {\textcolor{yellow}{\textbf{Annulé (Optionnel)}}}\\\hline
    {Acteur} & {\textcolor{yellow}{\textbf{Annulé (Optionnel)}}} & {\textcolor{yellow}{\textbf{Annulé (Optionnel)}}} & {\textcolor{yellow}{\textbf{Annulé (Optionnel)}}} & {\textcolor{yellow}{\textbf{Annulé (Optionnel)}}}\\\hline
    {Objet} & {\textcolor{yellow}{\textbf{Annulé (Optionnel)}}} & {\textcolor{yellow}{\textbf{Annulé (Optionnel)}}} & {\textcolor{yellow}{\textbf{Annulé (Optionnel)}}} & {\textcolor{yellow}{\textbf{Annulé (Optionnel)}}}\\\hline
    {Note} & {\textcolor{yellow}{\textbf{Annulé (Optionnel)}}} & {\textcolor{yellow}{\textbf{Annulé (Optionnel)}}} & {\textcolor{yellow}{\textbf{Annulé (Optionnel)}}} & {\textcolor{yellow}{\textbf{Annulé (Optionnel)}}}\\\hline
    {for} & {\textcolor{yellow}{\textbf{Annulé (Optionnel)}}} & {\textcolor{yellow}{\textbf{Annulé (Optionnel)}}} & {\textcolor{yellow}{\textbf{Annulé (Optionnel)}}} & {\textcolor{yellow}{\textbf{Annulé (Optionnel)}}}\\\hline
    {alt/else} & {\textcolor{yellow}{\textbf{Annulé (Optionnel)}}} & {\textcolor{yellow}{\textbf{Annulé (Optionnel)}}} & {\textcolor{yellow}{\textbf{Annulé (Optionnel)}}} & {\textcolor{yellow}{\textbf{Annulé (Optionnel)}}}\\\hline
    {ref} & {\textcolor{yellow}{\textbf{Annulé (Optionnel)}}} & {\textcolor{yellow}{\textbf{Annulé (Optionnel)}}} & {\textcolor{yellow}{\textbf{Annulé (Optionnel)}}} & {\textcolor{yellow}{\textbf{Annulé (Optionnel)}}}\\\hline
    {loop} & {\textcolor{yellow}{\textbf{Annulé (Optionnel)}}} & {\textcolor{yellow}{\textbf{Annulé (Optionnel)}}} & {\textcolor{yellow}{\textbf{Annulé (Optionnel)}}} & {\textcolor{yellow}{\textbf{Annulé (Optionnel)}}}\\\hline
    {opt} & {\textcolor{yellow}{\textbf{Annulé (Optionnel)}}} & {\textcolor{yellow}{\textbf{Annulé (Optionnel)}}} & {\textcolor{yellow}{\textbf{Annulé (Optionnel)}}} & {\textcolor{yellow}{\textbf{Annulé (Optionnel)}}}\\\hline
    
    \end{tabular}
    \\
    \newpage

    Tests de déplacement et de changement de couleur d’entités~:\\
    (Ne pas oublier que le déplacement d’une entité reliée à une relation doit faire suivre la relation)\\
    \begin{tabular}{|c|c|c|}\hline
    {Identifiant~:} & \multicolumn{2}{|l|}{TEST\_19} \\\hline
    {Identifiant de l’exigence~:} & \multicolumn{2}{|p{10cm}|}{MOD\_40 - MOD\_50} \\\hline
    \multicolumn{3}{|c|}{\textbf{Diagramme de cas d’utilisation}} \\\hline
    {\textbf{Entité}} & {Déplacer} & {Modifier la couleur} \\\hline
    {Note} & {\textcolor{green}{\textbf{Fait}}} & {\textcolor{green}{\textbf{Fait}}}\\\hline
    {Frontière} & {\textcolor{yellow}{\textbf{Annulé (Optionnel)}}} & {\textcolor{yellow}{\textbf{Annulé (Optionnel)}}}\\\hline
    {Acteur} & {\textcolor{green}{\textbf{Fait}}} & {\textcolor{green}{\textbf{Fait}}}\\\hline
    {Cas d’utilisation} & {\textcolor{green}{\textbf{Fait}}} & {\textcolor{green}{\textbf{Fait}}}\\\hline
    \multicolumn{3}{|c|}{\textbf{Diagramme de classe}} \\\hline
    {\textbf{Entité}} & {Déplacer} & {Modifier la couleur} \\\hline
    {Énumération} & {\textcolor{green}{\textbf{Fait}}} & {\textcolor{green}{\textbf{Fait}}}\\\hline
    {Classe abstraite} & {\textcolor{green}{\textbf{Fait}}} & {\textcolor{green}{\textbf{Fait}}}\\\hline
    {Interface} & {\textcolor{green}{\textbf{Fait}}} & {\textcolor{green}{\textbf{Fait}}}\\\hline
    {Classe} & {\textcolor{green}{\textbf{Fait}}} & {\textcolor{green}{\textbf{Fait}}} \\\hline
    {Note} & {\textcolor{green}{\textbf{Fait}}} & {\textcolor{green}{\textbf{Fait}}}\\\hline
    {Paquetage} & {\textcolor{yellow}{\textbf{Annulé (Optionnel)}}} & {\textcolor{yellow}{\textbf{Annulé (Optionnel)}}}\\\hline
    \multicolumn{3}{|c|}{\textbf{Diagramme de séquence}} \\\hline
    {\textbf{Entité}} & {Déplacer} & {Modifier la couleur} \\\hline
    {Acteur} & {\textcolor{yellow}{\textbf{Annulé (Optionnel)}}} & {\textcolor{yellow}{\textbf{Annulé (Optionnel)}}} \\\hline
    {Objet} & {\textcolor{yellow}{\textbf{Annulé (Optionnel)}}} & {\textcolor{yellow}{\textbf{Annulé (Optionnel)}}} \\\hline
    {Note} & {\textcolor{yellow}{\textbf{Annulé (Optionnel)}}} & {\textcolor{yellow}{\textbf{Annulé (Optionnel)}}} \\\hline
    {for} & {\textcolor{yellow}{\textbf{Annulé (Optionnel)}}} & {\textcolor{yellow}{\textbf{Annulé (Optionnel)}}} \\\hline
    {alt/else} & {\textcolor{yellow}{\textbf{Annulé (Optionnel)}}} & {\textcolor{yellow}{\textbf{Annulé (Optionnel)}}} \\\hline
    {ref} & {\textcolor{yellow}{\textbf{Annulé (Optionnel)}}} & {\textcolor{yellow}{\textbf{Annulé (Optionnel)}}} \\\hline
    {loop} & {\textcolor{yellow}{\textbf{Annulé (Optionnel)}}} & {\textcolor{yellow}{\textbf{Annulé (Optionnel)}}} \\\hline
    {opt} & {\textcolor{yellow}{\textbf{Annulé (Optionnel)}}} & {\textcolor{yellow}{\textbf{Annulé (Optionnel)}}} \\\hline
    \end{tabular}
    \\
    \newline

     \begin{tabular}{|r|p{5cm}|p{5cm}|}\hline
    {Identifiant~:} & \multicolumn{2}{|p{10cm}|}{TEST\_20} \\\hline
    {Identifiant de l’exigence~:} & \multicolumn{2}{|p{10cm}|}{MOD\_70} \\\hline
        {Objet~:} & \multicolumn{2}{|p{10cm}|}{Modifier le style d’une flèche} \\\hline
        {Action~:} & \multicolumn{2}{|p{10cm}|}{Modifier le style d’une flèche de toutes les manières possibles} \\\hline
        {Résultat attendu~:} & \multicolumn{2}{|p{10cm}|}{Le style de la flèche a été changé visuellement et respecte l’UML2} \\\hline
        {État~:} & {\textcolor{green}{\textbf{Fait}}} & {Le~: 03/05/2016 } \\\hline
    \end{tabular}
    \\
    \newline

    \begin{tabular}{|r|p{5cm}|p{5cm}|}\hline
    {Identifiant~:} & \multicolumn{2}{|p{10cm}|}{TEST\_21} \\\hline
    {Identifiant de l’exigence~:} & \multicolumn{2}{|p{10cm}|}{MOD\_110} \\\hline
        {Objet~:} & \multicolumn{2}{|p{10cm}|}{Inverser le sens d’une flèche} \\\hline
        {Action~:} & \multicolumn{2}{|p{10cm}|}{Inverser le sens d’une flèche de toutes les manières possibles} \\\hline
        {Résultat attendu~:} & \multicolumn{2}{|p{10cm}|}{Le sens de la flèche a été changé et respecte UML2} \\\hline
        {État~:} & {\textcolor{green}{\textbf{Fait}}} & {Le~: 03/05/2016 } \\\hline
    \end{tabular}
    \\
    \newline


    \begin{center}
    \begin{tabular}{|l|p{8cm}|r|c|}
        \hline\multicolumn{4}{|c|}{Modification d’un diagramme de cas d’utilisation} \\\hline
        {Identifiant~:} & \multicolumn{3}{|p{10cm}|}{TEST\_22} \\\hline
        {\textbf{Numéro}} & {\textbf{Description}} & {\textbf{Priorité}} & {\textbf{Testé et approuvé}}\\\hline
        {USE\_10} & {Modifier le texte d’une note} & {Indispensable} & {\textcolor{green}{\textbf{Fait}}}\\\hline
        {USE\_20} & {Modifier le nom d’une frontière. Ce nom est une chaîne de caractères quelconque.} & {Indispensable} & {\textcolor{green}{\textbf{Fait}}} \\\hline
        {USE\_30} & {Modifier le nom d’un acteur. Ce nom est une chaîne de caractères quelconque.} & {Indispensable} & {\textcolor{green}{\textbf{Fait}}} \\\hline
        {USE\_40} & {Modifier le texte d’un cas d’utilisation. Ce texte est une chaîne de caractères quelconque.} & {Indispensable} & {\textcolor{green}{\textbf{Fait}}} \\\hline
    \end{tabular}
\end{center}

\begin{center}
    \begin{tabular}{|l|p{8cm}|r|c|}
        \hline\multicolumn{4}{|c|}{Modification d’un diagramme de classes} \\\hline
        {Identifiant~:} & \multicolumn{3}{|p{10cm}|}{TEST\_23} \\\hline
        {\textbf{Identification}} & {\textbf{Description}} & {\textbf{Priorité}} & {\textbf{Testé et approuvé}}\\\hline
        {CLA\_10} & {Modifier le nom d’un objet} & {Indispensable} & {\textcolor{green}{\textbf{Fait}}} \\\hline
        {CLA\_20} & {Changer le type d’un objet} & {Indispensable} & {\textcolor{green}{\textbf{Fait}}} \\\hline
        {CLA\_30} & {Modifier un attribut} & {Indispensable} & {\textcolor{green}{\textbf{Fait}}} \\\hline
        {CLA\_40} & {Modifier une méthode} & {Indispensable} & {\textcolor{green}{\textbf{Fait}}} \\\hline
        {CLA\_50} & {Modifier la visibilité d’un attribut ou d’une méthode} & {Indispensable} & {\textcolor{green}{\textbf{Fait}}} \\\hline
        {CLA\_60} & {Modifier le texte d’une note} & {Indispensable} & {\textcolor{green}{\textbf{Fait}}} \\\hline
        {CLA\_70} & {Modifier le nom d’un paquetage} & {Indispensable} & {\textcolor{yellow}{\textbf{Annulé (Optionnel)}}} \\\hline
        {CLA\_80} & {Modifier un objet (son nom, ses attributs et ses méthodes) contenu par un paquetage} & {Indispensable} & {\textcolor{green}{\textbf{Fait}}} \\\hline
        {CLA\_90} & {Redimensionner un paquetage et un objet} & {Indispensable} & {\textcolor{green}{\textbf{Fait}}} \\\hline
    \end{tabular}
\end{center}

\begin{center}
    \begin{tabular}{|l|p{8cm}|r|c|}
        \hline\multicolumn{4}{|c|}{Modification d’un diagramme de séquence} \\\hline
        {Identifiant~:} & \multicolumn{3}{|p{10cm}|}{TEST\_24} \\\hline
        {\textbf{Numéro}} & {\textbf{Description}} & {\textbf{Priorité}} & {\textbf{Testé et approuvé}}\\\hline
        {SEQ\_10} & {Modifier le nom d’un acteur. Ce nom est une chaîne de caractères quelconque} & {Indispensable} & {\textcolor{yellow}{\textbf{Annulé (Optionnel)}}}\\\hline
        {SEQ\_20} & {Modifier le nom d’un objet} & {Indispensable} & {\textcolor{yellow}{\textbf{Annulé (Optionnel)}}}\\\hline
        {SEQ\_30} & {Modifier le texte d’une note} & {Indispensable} & {\textcolor{yellow}{\textbf{Annulé (Optionnel)}}}\\\hline
        {SEQ\_40} & {Modifier le type d’un cadre d’itération~: for, alt, ref, etc.} & {Indispensable} & {\textcolor{yellow}{\textbf{Annulé (Optionnel)}}}\\\hline
        {SEQ\_50} & {Modifier le texte contenu dans un cadre d’itération} & {Indispensable} & {\textcolor{yellow}{\textbf{Annulé (Optionnel)}}}\\\hline
        {SEQ\_60} & {Modifier les messages contenus dans un cadre d’itération} & {Indispensable} & {\textcolor{yellow}{\textbf{Annulé (Optionnel)}}}\\\hline
        {SEQ\_70} & {Modifier la ligne de temps d’un objet} & {Indispensable} & {\textcolor{yellow}{\textbf{Annulé (Optionnel)}}}\\\hline
    \end{tabular}
\end{center}


\section{Test des exigences fontionnelles}
\subsection*{Exigences fonctionnelles}
  L’exigence EXF\_10 n’a pas de test associé. L’exigence est triviale.\\ 
  TEST\_16 vérifie EXF\_30.\\
  TEST\_11 vérifie EXF\_50.
  
\subsection*{Test}
  \subsection*{L’ergonomie}
    \begin{tabular}{|r|p{5cm}|p{5cm}|}\hline
    {Identifiant~:} & \multicolumn{2}{|p{10cm}|}{TEST\_25} \\\hline
      {Identifiant de l’exigence~:} & \multicolumn{2}{|p{10cm}|}{EXF\_20} \\\hline
        {Objet~:} & \multicolumn{2}{|p{10cm}|}{L’ergonomie} \\\hline
        {Action~:} & \multicolumn{2}{|p{10cm}|}{Réalisation d'un sondage pour tester la facilité et l'efficacité de l'application} \\\hline
        {Résultat attendu~:} & \multicolumn{2}{|p{10cm}|}{Toutes les actions doivent être faciles d’utilisation.
                              On doit pouvoir faire n’importe quelle action en au plus 4 clics.
                              Manipuler un diagramme doit être facile.} \\\hline
    {Test supplémentaire~:} & \multicolumn{2}{|p{10cm}|}{On pourra faire tester les versions graphiques au client, mais aussi aux étudiants
                                sans leur expliquer comment fonctionne l’application. S’ils arrivent facilement et rapidement
                                à maîtriser le logiciel alors le test sera réussi. On prendra également leurs avis et leurs conseils.} \\\hline
        {État~:} & {\textcolor{green}{\textbf{Fait}}} & {Le~: 03/05/2016 } \\\hline
    \end{tabular}
    \\
    \newline

    \subsection*{UML2}
    \begin{tabular}{|r|p{5cm}|p{5cm}|}\hline
    {Identifiant~:} & \multicolumn{2}{|p{10cm}|}{TEST\_26} \\\hline
    {Identifiant de l’exigence~:} & \multicolumn{2}{|p{10cm}|}{EXF\_40} \\\hline
        {Objet~:} & \multicolumn{2}{|p{10cm}|}{UML2} \\\hline
        {Action~:} & \multicolumn{2}{|p{10cm}|}{Manipuler les diagrammes} \\\hline
        {Résultat attendu~:} & \multicolumn{2}{|p{10cm}|}{Toutes les actions possibles en UML2 doivent pouvoir être faites avec l’application} \\\hline
        {Test supplémentaire~:} & \multicolumn{2}{|p{10cm}|}{Lister toutes les actions possibles en UML2 avec le client et les effectuer sur le produit.} \\\hline
        {État~:} & {\textcolor{green}{\textbf{Fait}}} & {Le~: 03/05/2016 } \\\hline
    \end{tabular}
    \\
    \newline

     \subsection*{Entités de même nom}
    \begin{tabular}{|r|p{5cm}|p{5cm}|}\hline
    {Identifiant~:} & \multicolumn{2}{|p{10cm}|}{TEST\_27} \\\hline
    {Identifiant de l’exigence~:} & \multicolumn{2}{|p{10cm}|}{EXF\_60} \\\hline
        {Objet~:} & \multicolumn{2}{|p{10cm}|}{Entités de même nom} \\\hline
        {Action~:} & \multicolumn{2}{|p{10cm}|}{Nommer deux entités dans un même diagramme avec le même nom} \\\hline
        {Résultat attendu~:} & \multicolumn{2}{|p{10cm}|}{L’application prévient que ce n’est pas possible} \\\hline
        {État~:} & {\textcolor{green}{\textbf{Fait}}} & {Le~: 03/05/2016 } \\\hline
    \end{tabular}
    \\
    \newline

    \subsection*{Modulaire}
    \begin{tabular}{|r|p{5cm}|p{5cm}|}\hline
    {Identifiant~:} & \multicolumn{2}{|p{10cm}|}{TEST\_28} \\\hline
    {Identifiant de l’exigence~:} & \multicolumn{2}{|p{10cm}|}{EXF\_70} \\\hline
        {Objet~:} & \multicolumn{2}{|p{10cm}|}{Modulaire} \\\hline
        {Action~:} & \multicolumn{2}{|p{10cm}|}{Une fois la structure logicielle terminée, on doit pouvoir rajouter des fonctionnalités facilement en étendant le code} \\\hline
        {Résultat attendu~:} & \multicolumn{2}{|p{10cm}|}{L’extension du code doit être possible. Imaginer à la fin de la rédaction du DAL la façon dont nous
        pouvons étendre l’application pour implémenter des fonctionnalités telles que l’édition de nouveaux diagrammes.} \\\hline
        {État~:} & {\textcolor{green}{\textbf{Fait}}} & {Le~: 03/05/2016 } \\\hline
    \end{tabular}
    \\
    \newline

    \subsection*{Fonctionnel sur les ordinateurs de l’université}
    \begin{tabular}{|r|p{5cm}|p{5cm}|}\hline
    {Identifiant~:} & \multicolumn{2}{|p{10cm}|}{TEST\_28} \\\hline
    {Identifiant de l’exigence~:} & \multicolumn{2}{|p{10cm}|}{EXF\_80} \\\hline
        {Objet~:} & \multicolumn{2}{|p{10cm}|}{Fonctionnel sur les ordinateur de l'université} \\\hline
        {Action~:} & \multicolumn{2}{|p{10cm}|}{Tester l’application à l’université} \\\hline
        {Résultat attendu~:} & \multicolumn{2}{|p{10cm}|}{L’application doit parfaitement marcher} \\\hline
        {État~:} & {\textcolor{green}{\textbf{Fait}}} & {Le~: 03/05/2016 } \\\hline
    \end{tabular}
    \\
    \newline         

\section{Qualité}
Toutes les fonctionnalités doivent être faites rapidement sur l’application (moins d’une seconde pour chaque action).
Le reverse devra s’exécuter en moins de 30 secondes.

\section{Satisfaction du client}
Le but final de l’application est de satisfaire le client. Pour ce faire, les tests ici présents vérifient que l’application finale respecte bien tous les points de la STB.
Chaque cas d’utilisation doit être implanté.
Les tests ici présents doivent également tous être respectés. On veillera à ce qu’il ne reste plus de «~\textcolor{green}{\textbf{Fait}}~».
De plus, on effectuera des livraisons régulières au client (tous les mois) en s’assurant qu’il est parfaitement satisfait de ce qu’on a fait.
On mettra l’accent sur l’ergonomie de l’application et sur la qualité.

\end{document}