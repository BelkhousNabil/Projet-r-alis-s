\documentclass[hidelinks, 10pt,a4paper]{article}

\usepackage[english,francais]{babel}
\usepackage[utf8]{inputenc}
\usepackage{geometry}
\usepackage[T1]{fontenc}
\usepackage[pdftex]{graphicx}
\usepackage{adjustbox}
\usepackage{color}
\usepackage{setspace}
\usepackage{hyperref}
\usepackage[french]{varioref}
\usepackage{comment}
\usepackage{pgfgantt}


\usepackage{fancyhdr}
\pagestyle{fancy}

\renewcommand{\headrulewidth}{1pt}
\fancyhead[L]{}
\fancyhead[C]{\textbf{UML Reverse}}
\fancyhead[R]{\includegraphics[width=2cm]{img/universite-rouen.jpg}}

\title{\bfseries Document d'aide pour la reprise du projet\\Projet UML Reverse}
\geometry{hmargin=2.5cm,vmargin=3cm}

\begin{document}
\maketitle
\begin{center}
\begin{tabular}{ll}
  Version~: & 1.0\\[.5em]
  Date~: & \date{\today}\\[.5em]
  Rédigé par~: & Anthony \textsc{Godin}\\
               &\\[.5em]
  Relu par~:
               &\\
\end{tabular}
\end{center}

\newpage
\tableofcontents
\newpage

\section{Introduction}
\section{Modèle}
  \subsection{Projet}
  \subsection{Diagramme commun}
  \subsection{Diagramme de classe}
  \subsection{Diagramme de cas d'utilisation}
  \subsection{Parseur}
\section{Vue}

\end{document}
