\documentclass[a4paper,10pt]{article}
\usepackage[utf8]{inputenc}
\usepackage{rotating}
\usepackage{adjustbox}
\usepackage{lscape}
\usepackage{geometry}   %pour les marges
\usepackage{color} %pour la definition de nouvelles couleurs
\usepackage{graphicx} %pour ajouter des images
\usepackage[final]{pdfpages}
\usepackage[french]{varioref}
\usepackage[francais]{babel} % ``Franciser '' le document
\usepackage
  [colorlinks=false, urlcolor=red, breaklinks, pagebackref, citebordercolor={0 0 0},
  filebordercolor={0 0 0}, linkbordercolor={0 0 0}, pagebordercolor={0 0 0},
  runbordercolor={0 0 0}, urlbordercolor={0 0 0}, pdfborder={0 0 0}]
    {hyperref} % Ajouter le package des lien redirigeant sans les encadrer
\usepackage{eurosym}

%couleurs pour les morceaux de code
\definecolor{codegreen}{rgb}{0,0.6,0}
\definecolor{codered}{rgb}{1,0.1,0.2}
\definecolor{codegray}{rgb}{0.5,0.5,0.5}
\definecolor{codepurple}{rgb}{0.58,0,0.82}
\definecolor{backcolour}{rgb}{0.95,0.95,0.92}


%modificaion des marges
\geometry{hmargin=2.5cm,vmargin=3cm}


%opening
\title{Analyse des risques~: UML Reverse}

\begin{document}
\maketitle
\begin{center}
\begin{tabular}{ll}
  Version~: & 4.2\\[.5em]
  Date~: & \date{\today}\\[.5em]
  Rédigé par~: & Stephen \textsc{Cauchois}\\
               & Florian \textsc{Inchingolo}\\
               & Anthony \textsc{Godin}\\
               &\\[.5em]
  Relu par~:   & Florian \textsc{Inchingolo}\\
               & Nicolas \textsc{Meniel}\\
               &\\
\end{tabular}
\end{center}
\newpage
\vspace*{\stretch{3}}
Le document a pour but de nous préparer à affronter les risques potentiels du projet 
afin de ne pas se retrouver bloqués si l’un d’eux apparaît durant le développement.
Les risques seront revus après chaque réunion afin de déterminer les plus probables pour 
la suite du projet, et de réfléchir au plus vite aux solutions à apporter.
Afin de prévenir les risques liés aux technologies utilisées, des tests ont été effectués
pour s’assurer que les principales fonctionnalités du projet ne se retrouverait pas bloquées 
par des manques de ces technologies.
Les risques ont déjà été réévalués plusieurs fois et ceux qui sont finalement dans ce fichier sont 
triés par leurs impacts estimés sur le projet. Il n’y a donc pas à proprement parler d’identification
du risque le plus important car celui-ci ressort du tri effectué sur tous les risques.
\vspace*{\stretch{5}}
\newpage
\begin{landscape}
  \begin{tabular}{|p{1cm}|p{1.2cm}|p{2cm}|p{1.6cm}|p{3.5cm}|p{3.5cm}|p{3.5cm}|p{2cm}|p{1.3cm}|}
    \hline{\textbf{Rang}} & {\textbf{Risque}} & {\textbf{Catégorie}} & {\textbf{Interne/ Externe}} & {\textbf{Description}}
    & {\textbf{Conséquences}} &  {\textbf{Réponse}} & {\textbf{Probabilité (0 - 1)}} & {\textbf{Impact (1 - 5)}}\\\hline

    \hline
    {1} & {Majeur} & {Temporel} & {Interne/ Externe} & {Manque de temps dû aux autres projets universitaires} & {Accumulation du retard,
    impossibilité de finir l’application} & {Mise en place de créneaux de travail par semaine} & {1} & {4}\\\hline

    {2} & {Majeur} & {Temporel} & {Interne} & {Conflit sur la gestion du projet} & {Retard sur le projet, produit non conforme aux attentes du client} & {Spécification d’un maximum
    de choses et mise en place de réunions régulières pour se mettre d’accord} & {0,5} & {2,5}\\\hline
    {3} & {Majeur} & {Temporel} & {Interne} & {Algorithme de placement automatique des flèches}  & {Perte de temps importante}
    & {En cas de perte de temps trop importante, mise en place d’un placement manuel} & {0,6} & {2,5}\\\hline
    {4} & {Moyen} & {Temporel / Technique} & {Interne} & {Drag and drop ne répondant pas aux attentes} & {Perte de temps, impossibilité de déplacer les entités} &
    {En cas de perte de temps trop importante, mise en place d’un déplacement par coordonnées.} & {0,3} & {1,5}\\\hline
    {5} & {Majeur} & {Temporel / Organisation} & {Interne} & {Problème lors de l’utilisation du parseur Antlr} & {Perte de temps importante, impossibilité de faire du reverse engineering}
    & {En cas de conflit ou de perte de temps trop importante, création d’un parseur manuellement} & {0,4} & {1}\\\hline
    {6} & {Mineur} & {Temporel} & {Interne} & {Problème lors de l’intégration} & {Perte de temps importante} & {Architecture modulaire + intégration continue}
    & {0,4} & {1}\\\hline
    {7} & {Majeur} & {Technique / Temporel} & {Interne} & {Utilisation d’un nouveau framework~: JavaFX} & {Perte de temps pour se familiariser
    avec le framework et s’assurer que le framework réponde aux attentes} & {Formation avant de coder, recherche + ébauche en JavaFX. En cas de perte de temps trop importante, retour sur Swing
    } & {0,1} & {1}\\\hline


  \end{tabular}
\end{landscape}
\end{document}
